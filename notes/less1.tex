\lecture{01}{20 ago 25}{Conceitos Fundamentais}
\section{Experimento Stern-Gerlach}

%TODO: Escrever sobre o expretimento de Stern-Gerlach

\begin{equation}
	\ket{S_x;+} = \frac{1}{\sqrt{2}}\ket{S_z;+} + \frac{1}{\sqrt{2}}\ket{S_z;-}
\end{equation}
\begin{equation}
	\ket{S_x;-} = -\frac{1}{\sqrt{2}}\ket{S_z;+} + \frac{1}{\sqrt{2}}\ket{S_z;-}
\end{equation}

A componente não bloqueada que sai do segundo aparato ($SG \boldsymbol{\hat x}$)
na figura 1.c deve ser considerada como uma superposição de $S_z +$ e $S_z -$.
Essa é a razão que duas componentes emergem do terceiro aparato ($SG\boldsymbol{\hat z}$).

