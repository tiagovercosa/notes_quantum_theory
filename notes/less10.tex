\lecture{10}{13 out 25}{Propagadores e Integrais de Caminho de Feynman}

\section{As Integrais de Caminho de Feynman}
\subsection{Integrais de caminho como a soma sobre caminhos}
Trabalharemos com  $ \ket{x_1},\ket{x_2},...,\ket{x_n} $ para os \textit{kets}.
Queremos calcular a amplitude de probabilidade de uma partícula quântica
sair de  $ (x_1, t_1) $ para  $ (x_n,t_n) $. Para isso, vamos dividir o
intervalo de tempo em  $ n-1 $ partes.
$$
t_{n+1}-t_n = \Delta t = \frac{t_{n}-t_{i}}{n-1}
$$
\begin{multline*}
\braket{x_{n},t_{n} | x_{1},t_{1}}= \int dx_{n-1} \int dx_{n-2} \cdots \int dx_2 \\
\braket{x_{n},t_{n} | x_{n-1}, t_{n-1}}\braket{ x_{n-1}, t_{n-1} | x_{n-2}, t_{n-2}} \dots \braket{x_{2},t_{2} |x_{1},t_{1}}  
\end{multline*}

A integração é realizada nos infinitos caminhos que ligam  $ (x_1, t_1) $ a
$ (x_n, t_n) $, eses dois, fixos. A ação no caso clássico é dado por:
$$
S[x(t)] \triangleq \int_{t_0}^{t1} dt L(x,\dot{x}) \rightarrow S(n,n-1)=\int_{t_{n-1}}^{t_n}
dt L(x,\dot{x})
$$
que é a ação clássica.

A ação clássica é um funcional que dende da trajetória  $ S[x(t)] $.
Proposta de Feynman à luz da ideia de Dirac. Associar
$$
\exp{\left(\frac{i}{\hbar}\int_{t_1}^{t_2} dt L(x,\dot{x}\right)}
$$
com  $ \langle x_2, t_2|x_1, t_1 \rangle $.  Para uma trajetória completa de 
$ (x_1, t_1) $ a  $ (x_2, t_2) $

\begin{multline*}
	\prod_{n=2}^{n} \exp\left[\frac{i}{\hbar}S(n,n-1)\right] = \exp\left(
	\sum_{n=L}^{N}\frac{i}{\hbar}S(n,n-1)\right) = \\
	\exp\left[\frac{i}{\hbar}S(n,n-1)+S(N-1, N-2)+...\right] = 
	\exp \left[\frac{i}{\hbar}S(N,1)\right]
\end{multline*}

Para calcular  $ \langle x_n, t_n | x_1, t_1 \rangle $ precisamos integrar
sobre todas as trajetórias.

$$
\langle x_n, t_n | x_1, t_1 \rangle = \sum_{\substack{\text{``todos os} \\ \text{caminhos''}}}
\exp\left(\frac{i}{\hbar} S(N,1) \right)
$$

Na mecânica quântica falta o princípio de Hamilton, que nos mostra que
dentre todos os caminhos possíveis, um único relevante:  $ \delta S=0 $.
Podemos obter o limite clássico $ (\hbar -> 0) $ a partir da expressão de
Feynman?

Quando $ S/\hbar \gg 1 $:
$$
\sum_{\substack{\text{``todos os} \\ \text{caminhos''}}} \rightarrow \sum_{\delta S=0}
.$$
No limite clássico a "interferência destrutiva`` gera os infinitos caminhos
e passa a importar somente 1, o clássico
$$
  \delta S = 0.
$$

Consideremos a trajetória para qual  $ \delta S(N,1) = 0 $, com  $ (x_1,t_1) $ 
e  $ (x_N, t_N) $ fixos. As trejatórias vizinhas tem a mesma ação associado
em primeira ordem com interferência constutiva! Isso para os caminhos em um
tubo de largura  $ \hbar $. No limite  $ \hbar -> 0 $, este tubo colapsa para 
$ \delta S = 0 $,  que é a trajetória clássica!

%TODO FIGURA!!!

Escrevemos
$$
\langle x_n,t_n|x_{n-1},t_{n-1}\rangle = \frac{1}{W(\Delta t)}
\exp \left[\frac{i}{\hbar} S(n,n-1) \right]
$$
Os fatos  $ W^{-1}(\Delta t) $ aparece incialmente por razões de análise
dimensional. Vamos avaliar quem é  $ S(n,n-1) $, no limite  $ \Delta t -> 0 $ 
$$
S(n,n-1)=\int_{t_{n-1}}^{t_n}dt\left[ \frac{1}{2}m\dot{x}^2-V(x)\right]
\approx \Delta t \left[\frac{m}{2}\left(\frac{x_n-x_{n-1}}{\Delta t}\right)^2
-V\left(\frac{x_n+x_{x+1}}{2}\right)\right].
$$

Assumamos que  $ W(\Delta t) $ não depende do potencial. Escolhamos o 
potencial mais simples, onde  $ v=0 $ (partícula livre).
$$
S(n,n-1) = \Delta t \frac{m}{2}\left(\frac{x_n - n_{n-1}}{2}\right)^2
$$
e
$$
%TODO: colocar snippet para fechamento matemático
\langle x_n, t_n | x_1, t_1 \rangle = \frac{1}{W(\Delta t)} \exp \left\{
\frac{i}{\hbar} \frac{m(x_n-x_{n-1})^2}{2\Delta t}\right\}.
$$

Obs.:$ \langle x_n, t_n | x_{n-1}, t_{n-1} \rangle |_{t_n = t_{n-1}} = \delta(x_n - x_{n-1})$

$$
x_n = x_{n-1} + \xi \text{ e } t_n = t_{n-1} + \Delta t  
$$

$$
\langle x_{n-1} + \xi, t_{n-1} + \Delta t | x_{n-1}, t_{n-1} \rangle =
\frac{1}{W(\Delta t)} \exp\left(\frac{im\xi^2}{2\hbar \Delta t}\right)
$$

integrar sobre todos os caminhos

%TODO equação na imagem
$$
	\int_{-\infty}^{\infty} \braket{x_{n-1}+\xi, t_{n-1}+\Delta t | x_{n-1}, t_{n-1}} = \frac{1}{W(\Delta T)}
	\int_{-\infty}^{\infty} d \xi \exp \left( \frac{im \xi^2}{2\hbar \Delta t} \right) = \\ 
	\frac{1}{W( \Delta t)} \sqrt{\frac{2 \pi i \Delta t}{m}}
$$

$$
 1 = \frac{1}{W( \Delta t)} \sqrt{\frac{2 \pi i \Delta t}{m}}
$$

No limite $\Delta t\rightarrow 0 $ 
$$
\langle x_n, t_n|x_{n-1}, t_{n-1} \rangle = \sqrt{\frac{m}{2\pi i \hbar \Delta t}}
\exp\left[ \frac{i}{\hbar}S(n,n-1) \right]
$$
Amplitude infinitesimal para:

%TODO equação na imagem
...

\begin{itemize}
	\item Princípio da Superposição;
	\item Princípio da composição das amplitudes;
	\item Satisfaz o princípio da correspondência $( \hbar -> 0)$
	\item A expressão se reduz à equação de Schrödinger;
\end{itemize}

prova propagador de Feynman

$ t_n - t_{n-1} = \Delta t $ infinitesimal

$ x_n + x_{n-1} = \xi $ 
$$
\langle x_n, t_n | x_{n-1},t_{n-1} \rangle = \frac{m}{2\pi i\hbar \Delta t}
\exp\left\{\frac{i}{\hbar}\left(\frac{m}{2}\left(\frac{x_n-x_{n-1}}{\Delta t}\right)^2
	-\Delta t V\left( \frac{x_n + x_{n-1}}{2} \right)\right)\right\}
$$.