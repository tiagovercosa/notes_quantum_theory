\lecture{05}{15 out 25}{Evolução temporal e equação de Schrödinger}

\section{Evolução temporal e equação de Schrödinger}
\subsection{Operador de evolução tempotal}

Pergunta principal: Como o sistema muda por um deslocamento temporal  $t_{0}\rightarrow
t$? 

$$
\ket{\alpha,t_{0}} = \ket{\alpha} \xrightarrow{\hspace{2em}} \ket{\alpha,t_{0}; t}
.$$
No caso de translação, os \textit{kets} acima estão relacionados por meio do operador
$ \mathcal{U}(t,t_{0})$, denomidado \textbf{operador de evolução temporal}:
\begin{equation}
\ket{\alpha,t_{0}; t} = \mathcal{U}(t,t_{0})\ket{\alpha,t_{0}}.
\label{op_evotemp}
\end{equation}

A primeira propriedade importante é a unitariedade de  $ \mathcal{U}(t,t_{0}) $, que
vem da conservação da probabilidade. Suponhamos que em  $t_{0} $, o vetor de estado
esteja expandido em termos de autovetores de algum observável  $ A $ :
$$
\ket{\alpha,t_{0}} = \sum_{a'}c_{a'}(t_{0})\ket{\alpha'}
.$$
Do mesmo modo, num tempo posterior temos:
$$
\ket{\alpha,t_{0}; t} = \sum_{a'}c_{a'}(t)\ket{\alpha'}
.$$
Em geral, não se espera que o módulos dos coeficientes sejam iguais,
$$
  |c_{a'}(t)| \neq |c_{a'}(t_{0})|
.$$
No entanto, se o Hamiltoniano comuta com  $ A $, esses realmente serão iguais.

Se o \textit{ket} estava inicialmente normalizado em 1, ele deve permanecer normalizado
para totos os tempos posteriores. Desse modo, por exemplo, quando o spin  $ 1/2 $ precessiona
no eixo  $ xy $ ao passar por um campo magnético uniforme na direção  $ z $,
a probabilidade de observar  $S_{x}+ $ não é mais 1 em  $ t >t_{0} $; há uma
probalidade de se observar  $S_{x}+ $ e  $S_{x}- $. Mesmo assim, a soma das
probabilidades de se observar  $S_{x}+ $ e  $S_{x}- $ permanece igual a 1 durante
todo o tempo:
$$
\sum_{a'}|c_{a'}(t_{0})|^2 = \sum_{a'}|c_{a'}(t)|^2
.$$
Na notação de Dirac, tem-se
$$
  \braket{\alpha,t_{0} | \alpha,t_{0}} = 1 \Longrightarrow \braket{\alpha,t_{0}; t |
	\alpha,t_{0}; t} = 1
.$$

Este condição será satisfeita se tomarmos o operador como sendo unitário. Por
este motivo, teremos
$$
\mathcal{U}^\dag (t,t_{0})\mathcal{U}(t,t_{0}) = 1 
,$$
como sendo uma das propriedades funcamentais do operador  $ \mathcal{U} $, podendo
tomar a unitariedade como sinônimo de conservação de probabilidade.

Outra propriedade exigida para o operador  $ \mathcal{U} $ é a de composição,
que consiste em obter a evolução temporal de  $t_{0} $ a  $t_{2} $, por meio de uma
decomposição entre os tempos  $t_{0} $ a  $t_{1} $ e depois  $t_{1} $ a  $t_{2} $.
Desse modo podemos escrever, devendo ser lida da direita para a esqueda!
$$
\mathcal{U}(t_{2},t_{0}) = \mathcal{U}(t_{2},t_{1})\mathcal{U}(t_{1},t_{0}), \qquad
 (t_{2} >t_{1} >t_{0})
.$$

Também é conveniente considerar o operador evolução temporal infinitesimal
$ \mathcal{U}(t_{0} + dt,t_{0}) $:
$$
\ket{\alpha,t_{0};t_{0}+dt} = \mathcal{U}(t_{0}+dt,t_{0})\ket{\alpha,t_{0}}
.$$
Devido a continuidade o operador de evolução temporal deve ser igual ao operador
identidade quando  $ dt $ vai a zero:
$$
\lim_{dt \rightarrow 0} \mathcal{U}(t_{0}+dt,t_{0}) = 1
.$$
Observe que, isso é crucial para respeitar o caso em que o sistema não evolui no tempo,
retornar o próprio  $ \ket{\alpha,t_{0}} $ na equação \ref{op_evotemp}.

