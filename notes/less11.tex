\lecture{11}{20 nov 25}{Teoria do Momento Angular}

\section{Teoria do Momento Angular}

\subsection{Rotações e Revelações de Comutação de Momento Angular}

\subsubsection{Rotações infinitesimais versus rotações finitas}

Rotações em torno do mesmo eixo comutam, ao passo que aquelas em torno
de eixos diferentes não comutam. Para compreender o que é dito, vamos
usar a representação das rotações em três dimensões usando matrizes
3x3 reais e ortogonais.

Considere um vetor $\mathbf{V}$ de componentes $(V_x, V_y, V_z)$ e uma
matriz $R$ que representa uma rotação em torno do eixo $z$, onde ao
girarmos o vertor $\mathbf{V}$, as três componentes se transformam
em um outro conjunto de números $(V_x', V_y', V_z')$. As matrizes de
rotação estão relacionadas por uma matriz $\mathbf{R}$ 3x3 ortorgonal:

\begin{equation}
  \begin{pmatrix} V'_x \\ V'_y \\ V'_z \end{pmatrix} = \mathbf{R} \begin{pmatrix} V_x \\ V_y \\ V_z \end{pmatrix}
\end{equation}

\begin{equation}
    RR^T = R^TR = I,
\end{equation}

onde $T$ é a transposta. A propriedade 

\begin{equation}
    \sqrt{V_x^2 + V_y^2 + V_z^2} = \sqrt{V_x'^2 + V_y'^2 + V_z'^2}
\end{equation}

das matrizes ortogonais é automaticamente satisfeita.

